%%% PREAMBLE %%%

% Strict mode

\RequirePackage[l2tabu, orthodox]{nag} % Warn when using deprecated constructs

% Document class and packages

\documentclass[10pt,a4paper,parskip,fleqn]{scrartcl}
\usepackage[a4paper,vmargin={30mm},hmargin={30mm}]{geometry} % Page margins
\usepackage[ngerman]{babel} % New German hyphenation (multilingual support)
\usepackage[utf8x]{inputenc} % Unicode support
\usepackage{graphicx} % Graphics support
\usepackage{enumitem} % List spacing
\usepackage{nopageno} % No page numbers
\usepackage{chngpage} % Adjust page width
\usepackage{calc} % Calculations
\usepackage{color} % Colors
\usepackage{hyperref} % Hyperlinks

% Font configuration

\usepackage[sc]{mathpazo} % Use palatino font
\usepackage[T1]{fontenc} % Use correct font encoding
\usepackage[babel=true]{microtype} % Micro-typographic optimizations
\usepackage{setspace} % Line spacing
\addtokomafont{disposition}{\rmfamily} % Set palatino as heading font
\setstretch{1.3}
\setlist{nolistsep} % No spacing in lists


%%% TITLEPAGE %%%

\title{\Huge coredump rapperswil}
\subject{Informationen zum Verein}


%%% MAIN DOCUMENT %%%

\begin{document}

\begin{titlepage}

	\maketitle

	\vspace{2cm}

	\begin{adjustwidth}{-\oddsidemargin-1in}{-\rightmargin-1in}
		\includegraphics[width=\paperwidth]{soldering.jpg}

		\vspace{-12mm}

		\definecolor{light-gray}{gray}{0.85}
		\hfill {\scriptsize \color{light-gray} CC BY-NC-SA flickr.com/cbeas}
	\end{adjustwidth}

	\vfill

\end{titlepage}

\section{Der Verein}

Der Verein \textit{coredump} in Rapperswil-Jona wurde am 26. August 2013 von
fünf HSR-Studenten aus den Studienrichtungen Informatik und Elektrotechnik
gegründet. Das Ziel des Vereines ist der Betrieb eines Hackerspaces in
Rapperswil.

Ein Hackerspace ist ein Raum oder Gebäude, in dem sich an Wissenschaft,
Technologie oder digitaler Kunst (und vielen anderen Bereichen) Interessierte
treffen und austauschen sowie an gemeinsamen Projekten arbeiten können. Mehr
Informationen zu verschiedenen Hackerspaces weltweit gibt es auf
\url{http://hackerspaces.org/}.

Der Begriff \textit{Hacker} wird dabei gemäss der ursprünglichen Definition
verwendet. Er bezeichnet nicht etwa Kriminelle, welche in fremde Computernetze
eindringen, sondern Technikenthusiasten, welche sich kreativ und intensiv mit
Technologie auseinandersetzen.

Die zwei Ziele des Vereins gemäss Vereinsstatuten sind:

\begin{itemize}
		\item Bereitstellung der Infrastruktur zur Arbeit an nicht-kommerziellen
			technischen Projekten.
		\item Förderung des Informationsaustausches an regelmässigen Treffen,
			speziell im Bereich der Technologie.
\end{itemize}

In unserem Hackerspace möchten wir einerseits die Möglichkeiten und das Umfeld
zur Umsetzung von interessanten Projekten anbieten, andererseits auch den
Austausch von Wissen fördern. Dies kann regelmässige Treffen, öffentliche
Kurzvorträge oder auch Bildungsangebote und Workshops für interessierte
Aussenstehende beinhalten.

In verschiedenen Schweizer Städten (Zürich, Basel, Ostermundigen, St.  Gallen,
Genf, \ldots) gibt es bereits ähnliche Projekte. In der Linthgegend mit dem
Einzugsgebiet rund um Rapperswil gibt es jedoch noch keine vergleichbaren
Angebote. Diese Lücke möchten wir mit unserem Hackerspace füllen.


% \section{Die Gründungsmitglieder}
% 
% Wir sind fünf technikbegeisterte Studenten oder Absolventen der
% Studienrichtungen Informatik oder Elektrotechnik an der Hochschule für Technik
% Rapperswil. Wir möchten den kreativen Umgang mit Technik fördern und
% interessierten Personen den praktischen Einstieg in die Elektronik und
% Informatik ermöglichen.
% 
% Als Studenten hätten wir natürlich auch im Rahmen der Hochschule einen solchen
% Verein gründen können, allerdings wäre dieser dann nur für Studenten zugänglich
% und würde Absolventen sowie nicht-studentische Interessenten ausschliessen.
% Deshalb haben wir uns bewusst dafür entschieden, einen von der HSR unabhängigen
% Verein zu gründen.

\section{Mitglied werden?}

Egal wie viel Erfahrung du im Bereich der Informatik, Elektronik oder sonstwo
hast, bei uns kann jeder Mitglied werden der sich für kreativen Umgang mit
Technik interessiert und motiviert ist, Neues zu lernen.

Die Mitgliedschaft als Nichtverdiener kostet 10 CHF im Monat. Für Verdiener sind
es 20 CHF.

Weitere Informationen findest du auf \url{https://www.coredump.ch/} oder per
Email an \url{vorstand@coredump.ch}.

\end{document}
