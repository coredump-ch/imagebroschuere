%%% PREAMBLE %%%

% Strict mode

\RequirePackage[l2tabu, orthodox]{nag} % Warn when using deprecated constructs

% Document class and packages

\documentclass[10pt,a4paper,parskip,fleqn]{scrartcl}
\usepackage[a4paper,vmargin={25mm},hmargin={30mm}]{geometry} % Page margins
\usepackage[ngerman]{babel} % New German hyphenation (multilingual support)
\usepackage[utf8x]{inputenc} % Unicode support
\usepackage{graphicx} % Graphics support
\usepackage{enumitem} % List spacing
\usepackage{nopageno} % No page numbers
\usepackage{chngpage} % Adjust page width
\usepackage{calc} % Calculations
\usepackage{color} % Colors
\usepackage[colorlinks=true, urlcolor=black]{hyperref} % Hyperlinks
\usepackage{subfig} % Subfigures (für fotos)
\usepackage{float} % H placing specifier

\captionsetup[figure]{font=small,aboveskip=4pt,belowskip=8pt}

% Font configuration

\usepackage[sc]{mathpazo} % Use palatino font
\usepackage[T1]{fontenc} % Use correct font encoding
\usepackage[babel=true]{microtype} % Micro-typographic optimizations
\usepackage{setspace} % Line spacing
\addtokomafont{disposition}{\rmfamily} % Set palatino as heading font
\setstretch{1.3}
\setlist{nolistsep} % No spacing in lists

% Constants

\newcommand{\membercount}{29}


%%% TITLEPAGE %%%

\title{\Huge Coredump Rapperswil}
\subject{Sponsoring-Anfrage}


%%% MAIN DOCUMENT %%%

\begin{document}

\begin{titlepage}

	\maketitle

	\vspace{2cm}

	\begin{adjustwidth}{-\oddsidemargin-1in}{-\rightmargin-1in}
		\includegraphics[width=\paperwidth]{img/soldering.jpg}

		\vspace{-12mm}

		\definecolor{light-gray}{gray}{0.85}
		\hfill {\scriptsize \color{light-gray} CC BY-NC-SA flickr.com/cbeas}
	\end{adjustwidth}

	\vfill

\end{titlepage}

\section{Management Summary}

Auf dieser Seite finden Sie eine stichwortartige Zusammenfassung des gesamten
Dokumentes.

\subsection{Über Uns}

\begin{itemize}
	\item Wir sind ein Verein, welcher Technikbegeisterten in Rapperswil und
		Umgebung die technische Infrastruktur für nichtkommerzielle Projekte
		anbietet. Daneben verfolgen wir auch Bildungsprojekte wie den Ferienpass
		Rapperswil-Jona oder öffentliche Vorträge.
	\item Unser Vereinsraum bietet Arbeitsplätze, eine Elektronik-Werkstatt mit
		Lötstation und Messgeräten, einen 3D-Drucker, eine Bibliothek, eine Sitzecke
		mit Getränkekühlschrank und vieles mehr.
	\item Wir befinden uns im Sonnenhof Rapperswil auf 50m².
	\item Aktuell haben wir \membercount{} Mitglieder.
	\item Unser typisches Mitglied ist zwischen 25 und 30 Jahre alt, wobei wir
		auch jüngere und ältere Mitglieder haben. Viele der Mitglieder haben ein
		Elektrotechnik- oder Informatikstudium hinter sich, oder studieren aktuell
		noch.
	\item Besucher sind jederzeit willkommen. Wir treffen uns jeden Montag ab 20
		Uhr.
	\item Einige Projekte unserer Mitglieder: Farbige interaktive LED-Steuerungen
		für Outdoor-Festivals, orthopädische Alltagshilfen aus dem 3D-Drucker,
		Wassertemperatursensoren für den Obersee, selber entwickelte Open Source
		PCB-Design-Software, eine selbst gebaute Tesla-Spule, ein mithilfe von
		Drohnen rekonstruiertes 3D-Modell des Schloss Rapperswils, eine
		selbstgebaute Infrarot-Fernbedienung für Canon-Spiegelreflexkameras, und
		vieles mehr.
\end{itemize}

\subsection{Finanzierung}

\begin{itemize}
	\item Wir finanzieren uns primär über die Mitgliederbeiträge und private
		Gönner. Ein Nichtverdienermitglied bezahlt 10 CHF pro Monat, ein
		Verdienermitglied 30 CHF pro Monat und ein Fördermitglied 50 CHF pro Monat.
	\item Obwohl unser Verein stetig wächst, benötigen wir momentan noch weitere
		finanzielle Unterstützung von Sponsoren oder Gönnern -- primär zur Deckung
		der Fixkosten.
	\item Als Gegenleistung erhält der Sponsor einen Werbeplatz in unserem
		Vereinsraum. Darauf könnten Sie beispielsweise Ihre Firma vorstellen oder
		Jobangebote publizieren.
	\item Desweiteren erhält der Sponsor einen stets sichtbaren Link auf unserer
		Website (\href{https://www.coredump.ch/}{\texttt{coredump.ch}}) und wird in
		einem öffentlichen Blogpost sowie an jeder GV den Mitgliedern vorgestellt.
	\item Sponsoring-Pakete: Small (80 CHF im Monat) oder Large (120 CHF im
		Monat). Andere Beträge sind nach Absprache auch möglich.
\end{itemize}

Weitere Informationen finden Sie auf den folgenden Seiten sowie auf
\href{https://www.coredump.ch/}{\texttt{coredump.ch}}.

\newpage

\section{Über Uns}

Als erster Hackerspace\footnote{Mehr Informationen dazu gibt es auf
\url{http://hackerspaces.org/}} in Rapperswil-Jona bieten wir Computer- und
Technikbegeisterten seit August 2013 eine Plattform, in der sie ihr Wissen
weitergeben, sich austauschen und dazulernen können. Damit reihen wir uns in
eine weltweite Bewegung von tausenden von Hackerspaces, Makerspaces, FabLabs,
Techniklabors und ähnlichen Vereinen ein. Unser Angebot ist in der Region des
oberen Zürichsees bisher einzigartig.

Es geht dabei nicht um das <<Hacken von Computern>>, wie es in den Medien häufig
dargestellt wird, sondern gemäss der ursprünglichen Definition eines
Hackers\footnote{Mehr zur ursprünglichen Definition eines Hackers gibt es auf
Wikipedia: \url{https://de.wikipedia.org/wiki/Hacker}} um den kreativen Umgang
mit Technik und Wissen. Wir bieten Infrastruktur und KnowHow, um
nichtkommerzielle Projekte alleine oder in Kollaboration zu realisieren.

Unser zentral im Sonnenhof Rapperswil gelegener 50m²-Vereinsraum bietet aktuell
eine Elektronikwerkstatt mit Lötstation, Labornetzteil und Messgeräten, einen
3D-Drucker mit diversen Druckmaterialien, Arbeitsplätze für mitgebrachte
Laptops, WLAN mit Internetverbindung, ein Lager mit Elektronikbauteilen und
Entwicklungs-Kits, eine kleine Bibliothek mit Fachliteratur, einen Kühlschrank
und eine Mikrowelle für die Verpflegung vor Ort.

In unserem Hackerspace möchten wir einerseits die Möglichkeiten und das Umfeld
zur Umsetzung von interessanten technischen oder künstlerischen Projekten
anbieten, andererseits aber auch den Austausch von Wissen fördern. Dies kann
regelmässige Treffen, öffentliche Kurzvorträge oder auch Bildungsangebote und
Workshops für interessierte Aussenstehende (z.B. Ferienpass-Kurse für
Schulkinder) beinhalten.

\includegraphics[width=\textwidth]{img/raum.jpg}

\section{Projekte}

Damit Sie sich ein Bild unserer Aktivitäten machen können, drei unserer
aktuellen Projekte:

\textbf{Ferienpass im Computer- und Technikbereich}

Im November 2014 boten wir zum ersten Mal im Rahmen des Ferienpasses
Rapperswil-Jona einen Löt- und einen Programmierkurs für Kinder an. Diese -- wie
auch wir selbst -- waren davon begeistert: Wir erhielten 86 Anmeldungen und
konnten davon aus Platzgründen leider nur 16 berücksichtigen. Auch im Herbst
2015 und 2016 boten wir wieder Ferienpass-Kurse an, darunter ein neuer
3D-Druck-Kurs), mit sehr positivem Feedback. Wir freuen uns, auch in Zukunft
weiterhin Schulkinder aus Rapperswil-Jona mit Technik in Kontakt zu bringen und
dafür zu begeistern.

\textbf{3D-Druck}

Im Jahr 2015 durften wir unseren ersten 3D-Drucker in Betrieb nehmen,
erfolgreich finanziert über die Schweizer Crowdfunding-Plattform ``Wemakeit''.
Im Rahmen dieser Aktion haben wir unter anderem mit Hilfe eines Quadrokopters
einen 3D-Scan des Schlosses Rapperswil erstellt, diese Daten in Kooperation mit
der Hochschule für Technik HSR sowie dem Rapperswiler Start-Up ``Drei-De''
aufbereitet und ein druckfertiges Modell erzeugt.

\textbf{Wassertemperatursensoren Zürichsee}

Aktuell ist es für badefreudige Personen am Obersee schwierig, herauszufinden
wie warm oder kalt der See aktuell ist. Wir möchten deshalb in Kooperation mit
lokalen Vereinen sowie der Hochschule für Technik HSR den Obersee (und später
den Rest des Zürichsees) mit einem Netzwerk aus solarbetriebenen
Wassertemperatur-Sensoren ausstatten und diese Messdaten öffentlich über Mobile
Apps sowie auch über Programmier-Schnittstellen zugänglich machen.

\includegraphics[width=\textwidth]{img/tesla.jpg}

\section{Finanzierung}

Bisher finanzieren wir uns primär über die Mitgliederbeiträge. Ein
Nichtverdiener-Mitglied bezahlt bei uns 10~CHF im Monat, ein Verdiener-Mitglied
30~CHF und ein Fördermitglied 50~CHF. Aktuell beläuft sich unsere
Mitglieder-Anzahl auf \membercount{} Personen, die meisten davon sind entweder
Studenten an der HSR oder haben bereits eine technische Ausbildung hinter sich.
Daneben haben wir einige private Gönner.

Unser Verein wächst stetig. Im Mai sind wir aus Platzmangel aus unserem
vorherigen 16m²-Vereinsraum in neue Räumlichkeiten im Sonnenhof umgezogen. Um
unser Angebot in Zukunft jedoch weiterhin ausbauen zu können, reichen die
Mitgliederbeiträge nicht ganz aus. Deshalb sind wir auf der Suche nach Gönnern
und Sponsoren, welche uns dabei unterstützen könnten.

Wir möchten Ihnen folgende Sponsoring-Pakete vorschlagen:

\begin{itemize}
	\item Firmen-Sponsoring <<small>>: 80 CHF im Monat / 960 CHF im Jahr	
	\item Firmen-Sponsoring <<large>>: 120 CHF im Monat / 1440 CHF im Jahr	
\end{itemize}

Als Gegenleistung erhalten Sie als Sponsor folgende Möglichkeiten:

\begin{itemize}
	\item Einen Werbeplatz in unserem Vereinsraum: Ein A4-Blatt (<<small>>) bzw.
		ein A3-Blatt (<<large>>), auf dem Sie Ihre Firma vorstellen, Ihre Jobs
		ausschreiben, oder schlicht Ihr Firmenlogo positionieren können. Das Blatt
		werden wir in unseren Räumlichkeiten an der Wand befestigen, wodurch sich
		Mitglieder und Besucher über Ihre Firma informieren können.
	\item Einen stets sichtbaren Link auf unserer Website. Zudem publizieren wir
		einen Blogpost auf unserer Website, in dem wir Sie als Sponsor vorstellen.
	\item Wir werden Sie als Unterstützer an jeder GV den Mitgliedern nochmals
		kurz vorstellen.
\end{itemize}

Sponsoring soll sich lohnen. Den Gegenwert eines Sponsorings zu beziffern, ist
oft schwierig. Bei uns können Sie jedoch vergleichsweise günstig mit
potentiellen zukünftigen Fachkräften in Kontakt treten. Zudem unterstützen und
fördern Sie damit zugleich unsere Bemühungen, Kinder und Jugendliche (und
natürlich auch Erwachsene) für Technik zu begeistern und ihnen die Möglichkeit
geben, die Zukunft mitzugestalten statt nur zu konsumieren.

Auch wenn Sie sich nicht für ein Sponsoring entscheiden, dürfen Sie gerne an
einem Montag auf ein Bier oder Club Mate vorbeikommen und mit
Software-Entwicklern, Elektrotechnikern, HSR-/ETH-Studenten und Firmengründern
plaudern.

\newpage
\section{Weitere Informationen}

Weitere Informationen zu uns und unserem Verein finden Sie auf unserer Website:
\href{https://www.coredump.ch/}{\texttt{coredump.ch}}.

Für Rückfragen können Sie den Vorstand unter
\href{mailto:vorstand@lists.coredump.ch}{\texttt{vorstand@lists.coredump.ch}}
oder 079~728~93~96 kontaktieren. Gerne stellen wir uns auch mal persönlich bei
Ihnen vor.

\section{Fotos}

\begin{figure}[H]
\begin{tabular}{cc}
	\captionbox*{Laptop-Arbeitsplatz}{\includegraphics[width=0.5\textwidth]{img/laptop.jpg}} &
	\captionbox*{Eigene Leiterplatten ätzen}{\includegraphics[width=0.5\textwidth]{img/aetzen.jpg}} \\
	\captionbox*{3D-gedrucktes Schloss Rapperswil}{\includegraphics[width=0.5\textwidth]{img/schloss.jpg}} &
	\captionbox*{3D-Druck-Stand am Frühlingsfest 2016}{\includegraphics[width=0.5\textwidth]{img/fruehlingsfest.jpg}} \\
	\captionbox*{Ferienpass 2016: Löten}{\includegraphics[width=0.5\textwidth]{img/fepa.jpg}} &
	\captionbox*{Ferienpass 2016: 3D-Design}{\includegraphics[width=0.5\textwidth]{img/fepa2.jpg}} \\
\end{tabular}
\end{figure}

\end{document}
